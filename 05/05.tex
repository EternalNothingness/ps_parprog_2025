\documentclass[parskip]{scrartcl}

\usepackage{multirow}
\usepackage{minted}
\usepackage{xcolor}
\usepackage{enumitem}

\subject{Sheet 05}
\title{PS Parallel Programming}
\author{Patrick Wintner}
\date{\today}

\begin{document}
	\maketitle
	
	\section{Missing Flush Directives}
	Effects of missing flush directives are observed.
	\subsection{Source Code}
	\inputminted	[linenos]{c}{ex1/ex1.c}
	
	The program spawns two threads. Thread 0 does some  work (setting the value of the variable data) before setting a flag. Thread 1 should print the values of the flag and the variable after the other thread has finished his work.
	
	\subsection{Experiment Method}
	The experiment was done on the LCC3 cluster by calling 
	\begin{center}
		\colorbox{lightgray}{salloc --exclusive --tasks-per-node=1 --cpus-per-task=1 srun --pty bash}.
	\end{center} 
	Followed by calling 
	\begin{center}
		\colorbox{lightgray}{./main.sh}
	\end{center}		
	manually. The following scripts are involved in the experiment.
	\subsubsection{Main Script}
	\inputminted[linenos]{bash}{ex1/main.sh}
	
	\subsection{Experiment Results}
	The program neither terminates nor prints any output. This is probable the case because thread 1 does not fetch the updated value of flag from shared memory.  and is thus not able to leave the loop.
	
	\subsection{Discussion}
	The program does indeed require several flush directives, see the code below. Those are placed either after a variable (that is read in another thread) is written to or before a variable (that is written in another thread) is read. It does not require any atomic directives, because there is no variable to which is written in both threads.
	
	\inputminted[linenos]{bash}{ex1/ex1_improved.c}
	
	\section{Parallelising Code Snippets}
	\subsection{}
	\subsubsection{Original}
	\inputminted{c}{ex2/01.c}
	
	\subsubsection{Fixed}
	\inputminted{c}{ex2/01_fixed.c}
	The ordered construct is necessary to guarantee that the instructions are executed strictly in order.
	
	\subsection{}
	\subsubsection{Original}
	\inputminted{c}{ex2/02.c}
	
	\subsubsection{Fixed}
	\inputminted{c}{ex2/02_fixed.c}
	
	If threads do not wait after the first loop, it is possible that some elements of b are set with values of some elements of a that are not yet initialisized.
	
	\subsection{}
	\subsubsection{Original}
	\inputminted{c}{ex2/03.c}
	
	\subsubsection{Fixed}
	\inputminted{c}{ex2/03_fixed.c}
	
	Setting default to none requires the programmer to specify the storage attribute for each variable explicitly.
	\subsection{}
	\subsubsection{Original}
	\inputminted{c}{ex2/04.c}
	
	\subsubsection{Fixed}
	\inputminted{c}{ex2/04_fixed.c}
	
	The value of private variables is undefined on entry and exit of parallel regions.
	
	\subsection{}
	\subsubsection{Original}
	\inputminted{c}{ex2/05.c}
	
	\subsubsection{Fixed}
	\inputminted{c}{ex2/05_fixed.c}
	
	The reduction cause generates a local copy of sum and initialisizes it with 0. Updates occur on local copies (this would become a problem if sum were shared), which are afterwards combined to a single value.
	
	\subsection{}
	\subsubsection{Original}
	\inputminted{c}{ex2/06.c}
	
	\subsubsection{Fixed}
	\inputminted{c}{ex2/06_fixed.c}
	
	Avoids nesting of parallel regions. Only the loop variable of the parallel for loop is by default privat, therefore it is necessary to specify j as private. Another solution is to allow nesting of parallel regions by using the runtime routine omp\_set\_nested().
\end{document}
\documentclass[parskip]{scrartcl}

\usepackage{multirow}
\usepackage{minted}
\usepackage{xcolor}

\subject{Sheet 05}
\title{PS Parallel Programming}
\author{Patrick Wintner}
\date{\today}

\begin{document}
	\maketitle
	
	\section{Missing Flush Directives}
	Effects of missing flush directives are observed.
	\subsection{Source Code}
	\inputminted	[linenos]{c}{ex1/ex1.c}
	
	The program spawns two threads. Thread 0 does some  work (setting the value of the variable data) before setting a flag. Thread 1 should print the values of the flag and the variable after the other thread has finished his work.
	
	\subsection{Experiment Method}
	The experiment was done on the LCC3 cluster by calling 
	\begin{center}
		\colorbox{lightgray}{salloc --exclusive --tasks-per-node=1 --cpus-per-task=1 srun --pty bash}.
	\end{center} 
	Followed by calling 
	\begin{center}
		\colorbox{lightgray}{./main.sh}
	\end{center}		
	manually. The following scripts are involved in the experiment.
	\subsubsection{Main Script}
	\inputminted[linenos]{bash}{ex1/main.sh}
	
	\subsection{Experiment Results}
	The program neither terminates nor prints any output. This is probable the case because thread 1 does not fetch the updated value of flag from shared memory.  and is thus not able to leave the loop.
	
	\subsection{Discussion}
	The program does indeed require several flush directives, see the code below. Those are placed either after a variable (that is read in another thread) is written to or before a variable (that is written in another thread) is read. It does not require any atomic directives, because there is no variable to which is written in both threads.
	
	\inputminted[linenos]{bash}{ex1/ex1_improved.c}
	
	\section{Parallelising Code Snippets}
\end{document}
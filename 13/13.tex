\documentclass[parskip]{scrartcl}

\usepackage{minted}
\usepackage{pgfplots,filecontents}
\usepackage{enumitem}

\subject{Sheet 11}
\title{PS Parallel Programming}
\author{Patrick Wintner}
\date{\today}

\begin{document}
	\maketitle
	
	\section{Strength Reduction}
	\begin{enumerate}[label=\alph*)]
		\item Original: \inputminted[linenos,breaklines]{c}{ex1/a.c}
		
		reduced: \inputminted[linenos,breaklines]{c}{ex1/a_reduced.c}
		
		The transformation should be applied if a shift operation is cheaper than a multiplication.
		
		\inputminted[linenos,breaklines]{gas}{ex1/a.S}
		
		\item Original: \inputminted[linenos,breaklines]{c}{ex1/b.c}
		reduced: \inputminted[linenos,breaklines]{c}{ex1/b_reduced.c}
		
		The transformation should be applied if a shift operation, followed by a subtraction, is cheaper than a multiplication.
		
		\inputminted[linenos,breaklines]{gas}{ex1/b.S}
		
		\item Original: \inputminted[linenos,breaklines]{c}{ex1/c.c}
		reduced: \inputminted[linenos,breaklines]{c}{ex1/c_reduced.c}
		
		The transformation should be applied if two shift operations and an addition are cheaper than a multiplication.
		
		\inputminted[linenos,breaklines]{gas}{ex1/c.S}
		
		\item Original: \inputminted[linenos,breaklines]{c}{ex1/d.c}
		reduced: \inputminted[linenos,breaklines]{c}{ex1/d_reduced.c}
		
		The transformation should be applied if a shift operation is cheaper than a division.
		
		\inputminted[linenos,breaklines]{gas}{ex1/d.S}
		
		\item Original: \inputminted[linenos,breaklines]{c}{ex1/e.c}
		reduced: \inputminted[linenos,breaklines]{c}{ex1/e_reduced.c}
		
		The transformation can always be applied, because the transformation eleminates N/5 multiplications and otherwise replaces N/5 increment operations with the same number of ordinary additions.
		
		\inputminted[linenos,breaklines]{gas}{ex1/e.S}
		
		\item Original: \inputminted[linenos,breaklines]{c}{ex1/f.c}
		reduced: \inputminted[linenos,breaklines]{c}{ex1/f_reduced.c}
		
		The transformation can be applied if a division and N multiplications are cheaper than N divisions.
		
		\inputminted[linenos,breaklines]{gas}{ex1/f.S}
		
		\item Original: \inputminted[linenos,breaklines]{c}{ex1/g.c}
		reduced: \inputminted[linenos,breaklines]{c}{ex1/g_reduced.c}
		
		The transformation can be applied if defining an union, assigning a value to a variable, bitwise inversion and an Exor operation are cheaper than a floating point multiplication.
		
		\inputminted[linenos,breaklines]{gas}{ex1/g.S}
	\end{enumerate}
	
\end{document}
\documentclass[parskip]{scrartcl}

\usepackage{minted}

\subject{Sheet 08}
\title{PS Parallel Programming}
\author{Patrick Wintner}
\date{\today}

\begin{document}
	\maketitle
	
	\section{Compiler Dependence Analysis}
	The dependence analysis of compilers is examined.
	\subsection{Source Code}
	\inputminted	[linenos]{c}{ex1/analysis.c}
	
	\subsection{Makefile}
	\inputminted{make}{ex1/Makefile}
	
	\subsection{Discussion of Compiler Output}
	The full compiler output can be found in the file 08/ex1/output.log.
	
	% What information about dependence analysis can you find in the output? Can you identify any information pointing to successful or unsuccessful vectorization? Does the compiler perform any analysis beyond checking for dependencies and semantic correctness? What are your findings?
	
	Only the loop starting at line 10 has been successfully optimized. The last loop has not been optimized because the statement "clobbers memory" (which means here that the function call printf cannot be analyzed). The prelest loop starting at line 22 is analyzed using different vector modes and finally fails to vectorize the loop. The loop starting at line 10 has been optimized using 16 byte vectors. The loop starting at line 15 has not been vectorized because "more than one data ref in stmt" (probably because a and be could overlap). The loop starting at line 18 has also not been vectorized because the compiler previously recognized that the loop just performs a shift operation and has replaced it with an appropriate builtin-function.
	
	\section{Investigation of Code Snippets}
	\subsection{Safety of Parallelization}
	The safety of parallelization of the following code is examined.
	
	\inputminted	[linenos]{c}{ex2/01.c}
	
	It is possible that the arrays x and y overlap each other. One way to parallelize this manually is by using a temporary array z with the 1024*sizeof(double) and splitting the loop into two so that all read accesses from y (storing in z) are done in the first and all write accesses to x (reading from z) are done in the second. Both loops can run in parallel.
	
	The function cannot be parallelized by the compiler, because the compiler needs the \_\_restrict\_\_ keyword for both arguments, which implies that two pointers cannot point to overlapping memory regions.
	
	\subsubsection{Manually parallelized Loop}
	\inputminted	[linenos]{c}{ex2/01_par.c}
	
	\subsection{Loop Normalization}
	The following loop should be normalisized.
	
	\inputminted	[linenos]{c}{ex2/02.c}
	
	\subsubsection{Normalisized Loop}
	\inputminted	[linenos]{c}{ex2/02_normal.c}
	
	\subsection{Parallelizability}
	It is examined if the following loop is parallelizable.
	
	\inputminted	[linenos]{c}{ex2/03.c}
	
	The distance vector for all dependencies is $(1,0,-1)$ and therefore the corresponding direction vector is $(<,=,>)$. Thus the outmostloop is not parallelizable. The dependency of the first inner loop is loop-independent, therefore that loop can be parallelized.
	
	\subsubsection{Parallelized Loop}
	\inputminted[linenos]{c}{ex2/03_par.c}
\end{document}
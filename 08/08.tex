\documentclass[parskip]{scrartcl}

\usepackage{minted}

\subject{Sheet 08}
\title{PS Parallel Programming}
\author{Patrick Wintner}
\date{\today}

\begin{document}
	\maketitle
	
	\section{Compiler Dependence Analysis}
	The dependence analysis of compilers is examined.
	\subsection{Source Code}
	\inputminted	[linenos]{c}{ex1/analysis.c}
	
	\subsection{Makefile}
	\inputminted{make}{ex1/Makefile}
	
	% -ftree-vectorize: perform vectorization on trees.
	% -fopt-info-vec-all-internals:
		% -vec: enable dumps of all vectorization optimization
		% -all: detailed optimization information
		% -internals: even more detailed information
	\subsection{Discussion of Compiler Output}
	The full compiler output can be found in the file 08/ex1/output.log.
	
	% notes: bottom-up,
	
	\section{Investigation of Code Snippets}
	\subsection{Safety of Parallelization}
	The safety of parallelization of the following code is examined.
	
	\inputminted	[linenos]{c}{ex2/01.c}
	
	It is possible that the arrays x and y overlap each other. One way to parallelize this manually is by using a temporary array z with the 1024*sizeof(double) and splitting the loop into two so that all read accesses from y (storing in z) are done in the first and all write accesses to x (reading from z) are done in the second. Both loops can run in parallel.
	
	The function cannot be parallelized by the compiler, because the compiler needs the \_\_restrict\_\_ keyword for both arguments, which implies that two pointers cannot point to overlapping memory regions.
	
	\subsubsection{Manually parallelized Loop}
	\inputminted	[linenos]{c}{ex2/01_par.c}
	
	\subsection{Loop Normalization}
	The following loop should be normalisized.
	
	\inputminted	[linenos]{c}{ex2/02.c}
	
	\subsubsection{Normalisized Loop}
	\inputminted	[linenos]{c}{ex2/02_normal.c}
	
	\subsection{Parallelizability}
	It is examined if the following loop is parallelizable.
	
	\inputminted	[linenos]{c}{ex2/03.c}
	
	The distance vector for all dependencies is $(1,0,-1)$ and therefore the corresponding direction vector is $(<,=,>)$. Thus the outmostloop is not parallelizable. The dependency of the first inner loop is loop-independent, therefore that loop can be parallelized.
	
	\subsubsection{Parallelized Loop}
	\inputminted[linenos]{c}{ex2/03_par.c}
\end{document}
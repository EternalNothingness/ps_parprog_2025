\documentclass[parskip]{scrartcl}

\usepackage{multirow}
\usepackage{minted}
\usepackage{pgfplots,filecontents}

\subject{Sheet 06}
\title{PS Parallel Programming}
\author{Patrick Wintner}
\date{\today}

\begin{document}
	\maketitle
	
	\section{Monte Carlo}
	\label{ex1}
	The execution times of different implementations of the Monte Carlo method are measured.
	\subsection{Source Code}
	\subsubsection{Serial Implementation}
	\inputminted	[linenos]{c}{ex1/serial.c}
	\subsubsection{Parallel Implementation using a Critical Section}
	\inputminted	[linenos]{c}{ex1/critical.c}
	\subsubsection{Parallel Implementation using an Atomic Statement}
	\inputminted	[linenos]{c}{ex1/atomic.c}
	\subsubsection{Parallel Implementation using a Reduction Clause}
	\inputminted	[linenos]{c}{ex1/reduction.c}
	\subsection{Measurement Method}
	The measurement was done on the LCC3 cluster by calling \colorbox{lightgray}{sbatch job.sh serial 3}, \colorbox{lightgray}{sbatch job.sh critical 3}, \colorbox{lightgray}{sbatch job.sh atomic 3} and \colorbox{lightgray}{sbatch job.sh reduction 3}.
	
	The following scripts are involved in the experiment.
	\subsubsection{SLURM Script}
	\inputminted[linenos]{bash}{ex1/job.sh}
	\subsubsection{Benchmark Script}
	\inputminted[linenos]{bash}{ex1/benchmark.sh}
	
	\subsection{Experiment Results}
	\begin{tikzpicture}
		\begin{axis}[ymode=log]
			\addplot [color=blue, only marks, mark=o,]
			plot [error bars/.cd, y dir = both, y explicit]
 			table[x =x, y =y, y error =ey]{ex1/serial.dat};
 			\addplot [color=red, only marks, mark=o,]
			plot [error bars/.cd, y dir = both, y explicit]
 			table[x =x, y =y, y error =ey]{ex1/critical.dat};
 			\addplot [color=orange, only marks, mark=o,]
			plot [error bars/.cd, y dir = both, y explicit]
 			table[x =x, y =y, y error =ey]{ex1/atomic.dat};
 			\addplot [color=green, only marks, mark=o,]
			plot [error bars/.cd, y dir = both, y explicit]
 			table[x =x, y =y, y error =ey]{ex1/reduction.dat};
 			\legend{serial,critical,atomic,reduction}
		\end{axis}
	\end{tikzpicture}

	\subsection{Discussion}
	Unsurpisingly, there is no performance impact when using different numbers of threads for the serial implementation. The performance of the parallel implementation with an atomic statement is roughly equal to the performance of the parallel implementation and slightly detoriates with an increasing number of threads. Considering that only  the variable assignments to x and y (calling rand\_r) runs in parallel, while incrementing the counter is done serial, this is not really surprising. The performance of the parallel implementation with a critical section detoriates siginficantly when increasing the number of threads. Only the implementation using a reduction clause gains performance by increasing the number of threads. This meets the expections, because each thread has its own count, thus not slowing each other when incrementing the counter.
	
	\section{Mandelbrot}
	
	The execution time of an parallel implementation using openOMP of the mandelbrot set is measured.
	
	\subsection{Source Code}
	\inputminted	[linenos]{c}{ex2/mandelbrot.c}
	
	\subsection{Measurement Method}
	The measurement was done on the LCC3 cluster by calling \colorbox{lightgray}{sbatch job.sh mandelbrot 3}.
	
	The same scripts as in exercise \ref{ex1} are involved in the experiment.
	
	\subsection{Experiment Results}
	\begin{tikzpicture}
		\begin{axis}[]
			\addplot [color=blue, only marks, mark=o,]
			plot [error bars/.cd, y dir = both, y explicit]
 			table[x =x, y =y, y error =ey]{ex2/mandelbrot.dat};
 			\legend{mandelbrot}
		\end{axis}
	\end{tikzpicture}
	
	\subsection{Discussion}
	The execution time drops significantly when the number of threads is increased. Considering that threads are completely independent here, this behaviour is expected.
	
	\section{Loop Scheduling Methods}
	The impact of using different scheduling methods on openOMP implementations of the Hadamard product and the Mandelbrot set and the impact of increasing the size of the matrix on the execution time of the Hadamard product are observed.
	
	\subsection{Examples of Loop Scheduling Methods}
	\subsubsection{static}
	Divides the loop into equal-sized chunks.
	\subsubsection{dynamic}
	The iterations are broken into chunks of the specified size. When a thread finishes the execution of a chunk, the next chunk is assigned to that thread.
	\subsubsection{guided}
	Similar to dynamic, but decreases chunk sizes over time to reduce overhead during runtime.
	\subsubsection{runtime}
	The scheduling type and the chunk size is taken from  the OMP\_SCHEDULE environment variable.
	\subsubsection{auto}
	The scheduling type is left up the system.
	
	\subsection{Source Code}
	\inputminted	[linenos]{c}{ex3/hadamard.c}
	\inputminted	[linenos]{c}{ex3/mandelbrot.c}
	
	\subsection{Measurement Method}
	The measurement was done on the LCC3 cluster by calling \colorbox{lightgray}{sbatch job.sh benchmark\_scheduling mandelbrot 3}, \colorbox{lightgray}{sbatch job.sh benchmark\_scheduling hadamard 3}, \colorbox{lightgray}{sbatch job.sh benchmark\_threads hadamard 3} and \colorbox{lightgray}{sbatch job.sh benchmark\_matrix\_size hadamard 3}.
	
	The following scripts are involved in the experiment.
	\subsubsection{SLURM Script}
	\inputminted[linenos]{bash}{ex3/job.sh}
	\subsubsection{Benchmark Scripts}
	\inputminted[linenos]{bash}{ex3/benchmark_scheduling.sh}
	\inputminted[linenos]{bash}{ex3/benchmark_threads.sh}
	\inputminted[linenos]{bash}{ex3/benchmark_matrix_size.sh}
	
	\subsection{Experiment Results}
	\subsection{Scheduling Method}
	The number of threads is set to 12, the size of the matrix is $10.000^2$
	
	\begin{tikzpicture}
		\begin{axis}[
        		ybar,
        		xtick=data,
        		ymajorgrids=true,
        		xtick align=inside,
        		xticklabels={static, dynamic, guided, auto, runtime},
        		x tick label style={font=\normalsize, rotate=45, anchor=east}
        	]
        		\addplot table [x index=1,y index=2] {ex3/mandelbrot_scheduling.dat};
        		\addplot table [x index=1,y index=2] {ex3/hadamard_scheduling.dat};
        		\legend{mandelbrot, hadamard}
      	\end{axis}
	\end{tikzpicture}
	
	The scheduling method has no impact on the hadamard product. The performance of the implementation of the mandelbrot set is improved by dynamic and similar scheduling methods.
	\subsection{Number of Threads}
	The used scheduling method is static, the size of the matrix is set to $10.000^2$.
	
	\begin{tikzpicture}
		\begin{axis}[]
			\addplot [color=blue, only marks, mark=o,]
			plot [error bars/.cd, y dir = both, y explicit]
 			table[x =x, y =y, y error =ey]{ex3/hadamard_threads.dat};
		\end{axis}
	\end{tikzpicture}
	
	The best performance can be achieved  by setting the number of threads to four.
	
	\subsection{Size of Matrix}
	The used scheduling method is static, the number of threads is set to 12.
	
	\begin{tikzpicture}
		\begin{axis}[]
			\addplot [color=blue, only marks, mark=o,]
			plot [error bars/.cd, y dir = both, y explicit]
 			table[x =x, y =y, y error =ey]{ex3/hadamard_matrix_size.dat};
		\end{axis}
	\end{tikzpicture}
	
	The execution time increases linearily wwith the length of the matrix.
\end{document}
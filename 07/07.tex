\documentclass[parskip]{scrartcl}

\usepackage{multirow}
\usepackage{minted}
\usepackage{pgfplots,filecontents}

\subject{Sheet 07}
\title{PS Parallel Programming}
\author{Patrick Wintner}
\date{\today}

\begin{document}
	\maketitle
	
	\section{Parallelizing Loops}
	The dependencies and parallelisation posibilities of code snippets are analysized.
	\subsection{}
	\subsubsection{Serial}
	\inputminted	[linenos]{c}{ex1/a_ser.c}
	
	Statement S anti-depends (Write-After-Read) on itself: $S\delta^{-1} S$. Anti-dependencies can be eliminated through variable renaming.
	\subsubsection{Parallel}
	\inputminted	[linenos]{c}{ex1/a_par.c}
	\subsection{}
	\subsubsection{Serial}
	\inputminted	[linenos]{c}{ex1/b_ser.c}
	
	Statement S2 truly depends (Read-After-Write) on S1 and S3 truly depends on the last instance of S1: $S1\delta S2, S2\delta S3$. The depency is obviously not loop-carried, therefore the loop can be parallelized by making 'a' private within the loop.
	\subsubsection{Parallel}
	\inputminted	[linenos]{c}{ex1/b_par.c}
	\subsection{}
	\subsubsection{Serial}
	\inputminted	[linenos]{c}{ex1/c_ser.c}
	
	Statement S2 both truly and anti-depends on S1: $S1\delta S2, S1\delta^{-1} S2$. There is no dependence within the loops, therefore the loops themselves can be parallelized.
	\subsubsection{Parallel}
	\inputminted	[linenos]{c}{ex1/c_par.c}
	
	\section{Parallelizing more Loops}
	The dependencies of code snippets are analysized and the code snippets themselves are parallelized. The wall time of both the serial and parallel versions is measured.
	
	\subsection{Measurement Method}
	All measurements were done on the LCC3 cluster by calling \colorbox{lightgray}{sbatch job.sh $<$executable$>$ 3}, e. g. \colorbox{lightgray}{sbatch job.sh a\_ser 3} (3 is the number of measurements) with the number of loop iterations set to 100000000.
	
	The following scripts are involved in the experiment.
	\subsubsection{SLURM Script}
	\inputminted[linenos]{bash}{ex2/job.sh}
	\subsubsection{Benchmark Script}
	\inputminted[linenos]{bash}{ex2/benchmark.sh}
	\subsection{}
	\subsubsection{Serial}
	\inputminted	[linenos]{c}{ex2/a_ser.h}
	
	The zero-th instance of both statements S2 and S3 truly depends on S1: $S1\delta S2, S1\delta S3$. Furthermore, S2 has true loop-carried dependence on S3 and S3 has a true loop-carried dependence on itself: $S3\delta S2, S3\delta S3, S3\delta^{-1} S3$.
	\subsubsection{Parallel}
	\inputminted	[linenos]{c}{ex2/a_par.h}
	
	It is used that the different instances of statement S2 do not depend on each other and that there exist a recurrence formula  to determine 'factor'.
	\subsubsection{Experiment Results}
	\begin{tikzpicture}
		\begin{axis}[xlabel={number of threads}, ylabel={t/s}]
			\addplot [color=blue, only marks, mark=o,]
			plot [error bars/.cd, y dir = both, y explicit]
 			table[x =x, y =y, y error =ey]{ex2/a_ser.dat};
 			\addplot [color=red, only marks, mark=o,]
			plot [error bars/.cd, y dir = both, y explicit]
 			table[x =x, y =y, y error =ey]{ex2/a_par.dat};
 			\legend{serial,parallel}
		\end{axis}
	\end{tikzpicture}
	
	The walltime of the serial implementation is not affected by the number of threads, while the walltime of the parallel implementation decreases exponentially. Therefore both implementations behave as expected.
	\subsection{}
	\subsubsection{Serial}
	\inputminted	[linenos]{c}{ex2/b_ser.h}
	
	Statement S1 truly depends on S2: $S2\delta S1$. This dependency is loop-carried. This snippet can be parallelized by having two sequentially executed loops. The loops themselves are executed in parallel.
	\subsubsection{Parallel}
	\inputminted	[linenos]{c}{ex2/b_par.h}
	\subsubsection{Experiment Results}
	\begin{tikzpicture}
		\begin{axis}[xlabel={number of threads}, ylabel={t/s}]
			\addplot [color=blue, only marks, mark=o,]
			plot [error bars/.cd, y dir = both, y explicit]
 			table[x =x, y =y, y error =ey]{ex2/b_ser.dat};
 			\addplot [color=red, only marks, mark=o,]
			plot [error bars/.cd, y dir = both, y explicit]
 			table[x =x, y =y, y error =ey]{ex2/b_par.dat};
 			\legend{serial,parallel}
		\end{axis}
	\end{tikzpicture}
	
	The walltime of the serial implementation is barely affected by the number of threads (as expected), while the walltime of the parallel implementation is minimal when the number of threads is four (and not twelve), which is surprising.
	\subsection{}
	\subsubsection{Serial}
	\inputminted	[linenos]{c}{ex2/c_ser.h}
	
	Statement S4 has a control dependency on S3 and a data dependency on S2. The first instance of statement S4 has a data dependency on S1. This code snippet can be parallelized by replacing the loop by two sequential loops so that all statements within a loop are independent from each other. The control dependency is retained by wrapping the second loop in an if-instruction.
	\subsubsection{Parallel}
	\inputminted	[linenos]{c}{ex2/c_par.h}
	\subsubsection{Experiment Results}
	\begin{tikzpicture}
		\begin{axis}[xlabel={number of threads}, ylabel={t/s}]
			\addplot [color=blue, only marks, mark=o,]
			plot [error bars/.cd, y dir = both, y explicit]
 			table[x =x, y =y, y error =ey]{ex2/c_ser.dat};
 			\addplot [color=red, only marks, mark=o,]
			plot [error bars/.cd, y dir = both, y explicit]
 			table[x =x, y =y, y error =ey]{ex2/c_par.dat};
 			\legend{serial,parallel}
		\end{axis}
	\end{tikzpicture}
	
	The walltime of the serial implementation is not affected by the number of threads (as expected), while the walltime of the parallel implementation is minimal when the number of threads is four (and not twelve), which is surprising.
	
	\section{... And One More Loop to Parallelize}
	\subsection{Serial}
	\inputminted	[linenos]{c}{ex3/serial.c}
	\subsubsection{Distance and Direction Vectors}
	\begin{tabular}{|c|c|c|c|c|}
		\hline i & j & S: a[i+2][j-1] & S: a[i][j]\\
		\hline 0 & 1 & a[2][0]= & =a[0][1]\\
		\hline 0 & 2 & \textbf{a[2][1]=} & =a[0][2]\\
		\hline 0 & 3 & \textbf{a[2][2]}= & =a[0][3]\\
		\hline 1 & 1 & a[3][0]= & =a[1][1]\\
		\hline 1 & 2 & \textbf{a[3][1]=} & =a[1][2]\\
		\hline 1 & 3 & \textbf{a[3][2]=} & =a[1][3]\\
		\hline 2 & 1 & a[4][0]= & \textbf{=a[2][1]}\\
		\hline 2 & 2 & a[4][1]= & \textbf{=a[2][2]}\\
		\hline 2 & 3 & a[4][2]= & =a[2][3]\\		
		\hline 3 & 1 & a[5][0]= & \textbf{=a[3][1]}\\
		\hline 3 & 2 & a[5][1]= & \textbf{=a[3][2]}\\
		\hline 3 & 3 & a[5][2]= & =a[3][3]\\
		\hline
	\end{tabular}
	
	\begin{tabular}{|c|c|c|c|}
		\hline dependence relation & array element & distance vector & direction vector\\
		\hline $S[0][2]\delta S[2][1]$ & a[2][1] & $(2,-1)$ & $(<,>)$\\
		\hline $S[0][3]\delta S[2][2]$ & a[2][2] & $(2,-1)$ & $(<,>)$\\
		\hline $S[1][2]\delta S[3][1]$ & a[3][1] & $(2,-1)$ & $(<,>)$\\
		\hline $S[1][3]\delta S[3][2]$ & a[3][2] & $(2,-1)$ & $(<,>)$\\
		\hline
	\end{tabular}
	
	There are four loop-carried true data dependencies in the code snippet. All write accesses happen before all read accesses. Therefore the loop can be divided into two loops, so that both loops themselves can be executed in parallel. This solution only works because the number of iterations is low enough. A solution that scales is to only parallelize the inner loop.
	\subsection{Parallel}
	\inputminted	[linenos]{c}{ex3/parallel.c}
\end{document}
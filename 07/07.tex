\documentclass[parskip]{scrartcl}

\usepackage{multirow}
\usepackage{minted}
\usepackage{pgfplots,filecontents}

\subject{Sheet 07}
\title{PS Parallel Programming}
\author{Patrick Wintner}
\date{\today}

\begin{document}
	\maketitle
	
	\section{Parallelisizing Loops}
	The dependencies and parallelisation posibilities of code snippets are analysized.
	\subsection{}
	\subsubsection{Serial}
	\inputminted	[linenos]{c}{ex1/a_ser.c}
	
	Statement S anti-depends (Write-After-Read) on itself: $S\delta^{-1} S$. Anti-dependencies can be eliminated through variable renaming.
	\subsubsection{Parallel}
	\inputminted	[linenos]{c}{ex1/a_par.c}
	\subsection{}
	\subsubsection{Serial}
	\inputminted	[linenos]{c}{ex1/b_ser.c}
	
	Statement S2 truly depends (Read-After-Write) on S1 and S3 truly depends on the last instance of S1: $S1\delta S2, S2\delta S3$. The depency is obviously not loop-carried, therefore the loop can be parallelized by making 'a' private within the loop.
	\subsubsection{Parallel}
	\inputminted	[linenos]{c}{ex1/b_par.c}
	\subsection{}
	\subsubsection{Serial}
	\inputminted	[linenos]{c}{ex1/c_ser.c}
	
	Statement S2 both truly and anti-depends on S1: $S1\delta S2, S1\delta^{-1} S2$. There is no dependence within the loops, therefore the loops themselves can be parallelized.
	\subsubsection{Parallel}
	\inputminted	[linenos]{c}{ex1/c_par.c}
	
	\section{Parallelisizing more Loops}
	The dependencies of code snippets are analysized and the code snippets themselves are parallelized. The wall time of both the serial and parallel versions is measured.
	
	\subsection{}
	\subsubsection{Serial}
	\inputminted	[linenos]{c}{ex2/a_ser.h}
	\subsubsection{Parallel}
	\inputminted	[linenos]{c}{ex2/a_par.h}
\end{document}